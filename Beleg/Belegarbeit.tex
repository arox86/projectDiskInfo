\documentclass[german,10pt]{article}
\usepackage{babel}
\usepackage[utf8]{inputenc}
\begin{document}
\title{Software Technik Projekt: CrystalDiskInfo}
\author{Philipp Horländer, Konrad Münch}
\date{Berlin, \today}
\maketitle
\newpage

\tableofcontents
\newpage

\section{Einführung}
\subsection{Aufgabenstellung}


\subsection{Beschreibung der Software}
CrystalDiskInfo ist ein nützliches Programm für die Bewertung von HDDs und SSDs. Mit Hilfe der S.M.A.R.T. Technologie (Self-Monitoring, Analysis, Reporting Technology) werden detaillierte Daten, wie zum Beispiel Seriennummer, Temperatur, Lese- und Schreibzyklen, Fehlerrate beim Lesen und Schreiben, Speicher und Puffergrößen, als auch die gesamten Betriebsstunden der Festplatten, aus den Festplatten-Controllern ausgelesen.


Wie und welche dieser Daten angezeigt werden lässt sich einstellen. Beispielsweise lässt sich die Seriennummer ausblenden um auf Screenshots nicht aufzutauchen.

Auch lassen sich Graphen über ausgewählte Daten und Datenträger anzeigen.

Eine weitere nützliche Funktion für Admins ist es sich bei zu hohen Temperaturen der Festplatten benachrichtigen zu lassen. Die geschieht mit einen Warnton, als auch via E-Mail.

Wenn man möchte lassen sich auch Farbschemata ändern.



\section{Use-Case-Modell}

anzeigenDatenträgerSMART
\\anzeigenGraph





\section{Domänenmodell}



\section{Weitere Modelle}



\section{Geforderte Änderungen}



\section{Modellierung nach den Änderungen}




\section{Beschreibung der Implementierung}



\section{Fazit}
Nach anfänglichen Schwierigkeiten haben wir CrystalDiskInfo dann auf allen unseren Windows Maschinen 



\end{document}