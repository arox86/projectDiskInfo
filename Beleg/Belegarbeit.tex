\documentclass[german,10pt]{article}
\usepackage{babel}
\usepackage[utf8]{inputenc}
\usepackage{graphicx}
\usepackage{ textcomp }
\usepackage{ tipa }
\usepackage[center]{caption}
\usepackage{float}


\begin{document}
\title{ {\huge Softwaretechnik Projekt: \\ Crystal Disk Info} \\{\large Computer Engineering, Wintersemester 2019/2020} \\Neue Fassung für das Sommersemester 2020} 
\author{Philipp Horländer, 566045\\Konrad Münch, 565929 \\ \\Dozent: Prof. Dr. Bauer}


\date{Berlin, \today}

\maketitle
\newpage

\tableofcontents
\cleardoublepage

\listoffigures
\newpage


\section{Einführung}
Im Rahmen der Lehrveranstaltung Softwaretechnik, des Studiengangs Computer Engineering an der HTW-Berlin, werden Inhalte zur Softwarerchitektur, Modellierung, Abstraktion sowie Anforderungen
an Software gelehrt. Des Weiteren werden verschiedene Notationen zur Darstellung und funktionalen 
Beschreibung vorgestellt. Ein weiterer Bestandteil des Moduls ist es, den praktischen Umgang mit den o.g. Lehrinhalten in einer praktischen Arbeit zu verbinden.
\subsection{Aufgabenstellung}

Es galt sich eine Open Source Software auszusuchen und diese zu ''Reverse-Engineeren'' und abzuändern. Dafür sollten verschiedenen Modelle der Software angefertigt werden. Beispiele dafür sind Use-Case-, Domänen- und ggf. weitere Modelle, dies sollte mit einer Belegarbeit dokumentiert werden.\\\\

\subsection{Visionen und Ziele des Originalprojektes}
\begin{itemize}
\item \textbf{/VO1/} Crystal Disk Info soll für Microsoft Betriebsysteme ab Windows XP verfügbar sein.
\item \textbf{/ZO2/} Crystal Disk Info soll die S.M.A.R.T. Technologie nutzen um die Festplattendaten kompatibler Laufwerke über eine bestimmte Zeitperiode auslesen zu können .
\item \textbf {/ZO3/} Crystal Disk Info soll über den Gesamtzustand der Festplatte informieren und 
bei niedrieger Lebenserwartung des Laufwerkes warnen.

\item \textbf {/ZO4/} Crystal Disk Info soll Diagramme darstellen können um die Verläufe der Werte
von \textbf {/ZO3/} grafisch darstellen zu können.

\end{itemize}


\subsection{Beschreibung der Software}

''Crystal Disk Info'' ist ein nützliches Open-Source Programm für die Bewertung und Überwachung von internen als auch externen HDDs und SSDs. Mit Hilfe der S.M.A.R.T. - Technologie (Self-Monitoring, Analysis, Reporting Technology) werden detaillierte Daten, wie zum Beispiel Seriennummer, Temperatur, Lese- und Schreibzyklen, Fehlerrate beim Lesen und Schreiben, Speicher und Puffergrößen, als auch die gesamten Betriebsstunden der Festplatten, aus den Festplatten-Controllern ausgelesen. 
Diese sind auf dem Controller-Chip gespeichert und werden in einem einstellbaren Intervall von der Software ausgelesen. Die entsprechenden Daten werden auf dem jeweiligen Host-System in Form von CSV-Dateien abgelegt.
\\\\
Wie und welche Daten angezeigt werden lässt sich einstellen. Beispielsweise lässt sich die Seriennummer ausblenden, um auf Screenshots nicht aufzutauchen.
\\\\
Auch lässt sich ein Diagramm über ausgewählte, gespeicherte Daten sowie Datenträgerwerte in Abhängigkeit der Zeit anzeigen.
\\\\
Eine weitere nützliche Funktion für Administratoren ist es, sich bei zu hohen Temperaturen oder einer geringen Lebenszeit der Festplatten benachrichtigen zu lassen. Am lokalen Rechner wird beim Auftreten einer Warnung eine Melodie abgespielt. Des Weiteren ist es möglich,
eine Warnung mit einer E-Mail automatisch zu versenden.
\\\\
Wenn gewünscht lassen sich auch Farbschemata der Buttons und Anzeigen ändern.\\\\


\subsection{Programm vor Änderung}
\begin{figure}[H]
	\centering
	\includegraphics [width=0.85\textwidth]{Bilder/before} 
	\caption{Screenshot Crystal Disk Info}
	\label{fig:Screenshot1}
\end{figure}\
\section{Aufbau des Programmes}
Um die gesammmte Architektur von Crystal Disk Info zusammhängend erfassen zu können, ist es relevant die Abschnitte Programmiersprache, Quellcode und Ordnerstruktur näher zu betrachten.
Die Erläuterungen in den folgenden Unterabschnitten betrachten jedoch nur, die wesentlichen
Bereiche welche für die hier geforderte Änderung des Programms relevant sind.
\\\\
\subsection{Programmiersprache(n)}

Crystal Disk Info wurde von Noriyuki Miyazaki in der Sprache C++ für Microsoft Windows Betriebssysteme programmiert.
Zur Darstellung der grafischen Benutzeroberfläche wurde die Microsoft Foundation Classes kurz MFC\footnote{MFC ist eine von Microsoft seit 1992 entwickelte Bibliothek zur Erstellung von Windows Desktop Anwendungen.} verwendent. Des Weiteren wurde für die Realisierung der Darstellung der Graphen, die JAVA -Bibliothek ''FLOT''  verwendet.\\\\
Zudem werden noch HTML Dateien genutzt, welche zur Darstellung der Graphen dienen.
Durch die Diagnosefunktion von GitHub lies sich feststellen, dass C++, mit 94,7\%, die meist verwendete Programmiersprache in Crystal Disk Info ist.
\\

\subsection{Übersicht des Quellcodes}

Zum Anfang des Projekts, wurden von uns die Analyse-Tools Sourcetrail und Doxygen genutzt, um eine Übersicht vom Umfang und Beschaffenheit des Quellcodes zu bekommen. Doxygen erwies sich als zu mächtig und umfangreich für unsere Zwecke. Die erzeugte RTF-Datei umfasste zwar jegliche Art von Diagrammen, jedoch war die Datei mit über 32500 Wörtern nicht praktikabel einsetzbar. 
\\\\
Sourcetrail erwies sich hingegen als komfortabler und übersichtlicher. Wir konnten mit Hilfe dieses Tools einfacher den Aufbau und die Beschaffenheit der Klassen und Funktionen erfassen.
\\
\begin{figure}[H]
	\centering
	\includegraphics [width=0.65\textwidth]{Bilder/sourceTrail} 
	\caption{SourceTrail Übersicht}
	\label{fig:SourceTrailU}
\end{figure}\
\\
Insgesamt umfasst der Code 34207 Zeilen an Code in 65 Dateien. Wie unter Punkt 2.1 schon erwähnt wurde MFC für den Aufbau des Programmes als Grundlage genutzt.
Durch die Analyse mit Sourcetrail konnten wir uns einen Überblick, über die wichtigste Ursprungsklasse von der alle GUI Elemente abhängen verschaffen. Aus der Klasse CWinApp werden alle erstellten Klassen wie Dialogfenster und Fensterfunktionen abgeleitet.
\\
Aufgrund der Größe des Programms, beschränken wir uns zunächst nur auf die Klasse CWinApp  und der des Hauptdialogs CDiskInfoApp (siehe Abbildung \ref{fig:CWinApp}).
Letztere ist die von dem Autor erstellte Elternklasse aus der sich alle weiteren Dialoge des GUI ableiten.\\\\
\\\\

\begin{figure}[H]
	\centering
	\includegraphics [width=1\textwidth]{Bilder/cwinapp}
	\caption{SourceTrail Übersicht von CWinApp und CDiskInfoApp}
	\label{fig:CWinApp}
\end{figure}  

\subsection{Ordnerstruktur}

Das Programm erzeugt als Root-Verzeichnis den Ordner ''Marginality''. Dieser enthält die EXE-Datei und  die Ordner CdiResource sowie Smart, zu sehen in Abbildung \ref{fig:Dir}. 
\\
CdiResource enthält wichtige externe Dateien, wie Sprachdateien, Themen, die JavaScript-Dateien für den Graphen sowie das Programm AlertMail welches genutzt wird um Warnungen via E-Mail zu versenden.
\\\\
Der Ordner ''Smart'' enthält die durch Crystal Disk Info ermittelten Werte der verbauten Festplatten im Host-System. Diese Daten werden vom GUI und der JavaScript-Bibliothek FLOT genutzt, um die Werte im Frontend des Programmes darzustellen. 
\\

\begin{figure}[H]
	\centering
	\includegraphics [width=0.85\textwidth]{Bilder/all}
	\caption{Ordner Struktur im Programm (nicht Source-Code)}
	\label{fig:Dir}
\end{figure} \
\\
Ohne diesen Aufbau ist das Programm  nicht lauffähig.
Das GIT-Repository von Crystal Disk Info enthält nicht den vollständigen Ordneraufbau. Es fehlt der Ordner CdiResource, dies führt nach einem Compiler-Durchlauf zu Laufzeitfehlern.
Bei der Installation mittels Setup Wizard werden alle nötigen Verzeichnisse erstellt.
\\\\
\section{Modellierung}
Zur formalen Darstellung des Funktionsumfangs von Crystal Disk Info, werden unter diesem Abschnitt 
zwei Notationen verwendet. Das sogenannte Use Case Diagramm welches in Abbildung 5 dargestellt ist, dient als Abstraktion um aufzuzeigen welche Möglichkeiten, dem Nutzer zur Verfügung stehen. Es informiert zudem über weitere Relationen wie z.B. zu externen Aktoren. 
\\\\
Des Weiteren wird mit dem Domänenmodell eine Übersicht der Klassen und Funktionen des Programmes gegeben. Unter dem Abschnitt 3.2 in Abbildung 6, befindet sich die relevanten Informationen.

\subsection{Use-Case-Modell}

\begin{figure}[H]
	
	\includegraphics [width=1\textwidth]{Bilder/UseCase1}
	\caption{Use-Case-Modell}
	\label{fig:UseCase1}
\end{figure}  


\subsubsection{anzeigenDatenträgerInformation}

1. Das Programm ermittelt die verbauten Festplatten.\\
2. Das Programm liest SMART-Daten jedes Festplatten-Controllers aus (\textbf{/ZO2/}).\\ 
3. Das Programm zeigt die ermittelten Werte im GUI  an.  


\subsubsection{einstellenRateAktualisierung}

1. Der Nutzer wählt ein Zeitintervall\\
2. Das Programm liest und speichert die Werte im gewählten Intervall (\textbf{/ZO2/}).


\subsubsection{anzeigenGraph}

1. Der Nutzer wählt die Funktion Graph anzeigen.\\
2. Das Programm öffnet ein neues Fenster.\\
3. Der Nutzer wählt eine Datenreihe zur Anzeige aus.\\
4. Das Programm zeigt einen Graph mit gewählten Datenreihen an (\textbf{/ZO4/}).
\\\\
2a. Der Nutzer wählt eine andere Datenreihe aus.\\
2a.1. Das Programm zeigt die andere Datenreihe an (\textbf{/ZO4/}).
\\\\
2b. Der Nutzer wählt Festplatten an/ab.\\
2b.1. Das Programm zeigt die gewählten Festplatten-Datenreihen an (\textbf{/ZO4/}).


\subsubsection{einstellenBenachrichtigung}

1. Der Nutzer wählt Einstellen einer Benachrichtigung aus.\\
2. Das Programm wartet Eingabe der benötigten Daten des Benutzers.\\
3. Der Nutzer gibt nötigen Daten ein und bestätigt.


\subsubsection{benachrichtigenNutzer}

1. Das Programm stellt einen kritischen Wert von Temperatur oder Lebensdauer einer Festplatte fest (\textbf{/ZO3/}).
\\
2. Das Programm erstellt eine E-Mail und versendet diese an die eingestellte Adresse.
\\\\
2a Der Nutzer schließt das Fenster.\\
2a.1 Das Use-Case endet erfolglos.


\subsubsection{auswahlTheme}

1. Der Nutzer wählt Theme ändern aus.\\
2. Das Programm zeigt dem Nutzer eine Auswahl an.\\
3. Der Nutzer wählt zwischen Verschiedenen Farbdarstellungen.\\
4. Das Programm lädt die ausgewählte Darstellung.
\\\\
(2-3)a Der Nutzer bricht die Auswahl ab.\\
(2-3)a.1 Das Use-Case endet erfolglos.
\\\\
4\textdoublevertline a. Das Programm stellt fest, dass benötigte Dateien fehlen:\\
4\textdoublevertline a.1. Das Programm zeigt eine Fehlermeldung an.\\
4\textdoublevertline a.2. Das Use-Case endet erfolglos an.


\subsubsection{auswahlSprache}

1. Der Nutzer wählt Sprache ändern aus.\\
2. Das Programm zeigt dem Nutzer eine Auswahl an.\\
3. Der Nutzer wählt zwischen verschiedenen Sprachen.\\
4. Das Programm lädt die ausgewählte Sprache.
\\\\
(2-3)a Der Nutzer bricht die Auswahl ab.\\
(2-3)a.1 Das Use-Case endet erfolglos.
\\\\
4\textdoublevertline a. Das Programm stellt fest das benötigte Dateien fehlen:\\
4\textdoublevertline a.1. Das Programm zeigt eine Fehlermeldung an.\\
4\textdoublevertline a.2. Das Use-Case endet erfolglos an.


\subsection{Domänenmodell}

Das in Abbildung \ref{fig:DomMod0} gezeigte Domänenmodell stellt die Funktion vor der Änderung des Programmes dar. Zur Übersicht wurde nur der Ausschnitt des gesamten Programmes betrachtet, der für die spätere Änderung relevant ist.\\

\begin{figure}[H]
	\centering
	\includegraphics [width=1.1\textwidth]{Bilder/DomaenenModell_0}	
	\caption{Domänenmodell: Funktionsweise}
	\label{fig:DomMod0}
\end{figure}


\section{Geforderte Änderungen}

\subsection{Visionen und Ziele der Änderung}
\begin{itemize}
\item \textbf{/ZÄ1/} Crystal Disk Info soll über einen neuen Diagrammtyp verfügen, welcher 
drei Parameter auf einem zweidimensionalen Graphen abbilden kann.

\item \textbf {/ZÄ2/} Crystal Disk Info soll sekündlich die Aktualisierung der ausgelesenen
Werte ermöglichen.
\\\\


\end{itemize}
Die Änderung von Crystal Disk Info sollte die Darstellung eines neuen Diagrammtyps (\textbf{/ZÄ1/}) ermöglichen. Das sogenannte ''Bubble Diagramm'' sollte in die Auswahlfunktion der Diagramme implementiert werden.
\\
Ein ''Bubble Diagramm'' soll die Korrelation von zwei Datenreihen darstellen, in Form von Punkten auf einem Koordinatensystem.
Diese werden durch eine dritte Abhängigkeit ergänzt welche durch die Größe der Punkte (Radius) dargestellt wird.
\\\\
Ein besprochene Umsetzungsidee war die Darstellung von Temperatur auf der X-Achse zu geschriebenen Daten auf der Y-Achse, welche nach einer Zeitperiode ihre Größe verändern. Zudem sollten noch andere Daten mit einander verglichen werden.
\\\\
Nach Absprache mit dem Dozenten Herrn Prof. Dr. Baar wurde die Art der Implementierung vereinfacht. Grund hierfür war die Komplexität des Aufbaus der Darstellungsfunktion im Programm. 
\\
Eine weitere Änderung sollte darin bestehen, die Aktualisierungszeit (\textbf{/ZÄ2/}) der Datenpunkte auf eine Sekunde hinzuzufügen.
\\\\

\section{Modellierung nach den Änderungen}

\subsection{Use-Case-Modell}

Änderungen gab es an zwei Use-Cases.


\subsubsection{anzeigenDiagramm}

 Das Programm generiert eine HTML-Datei und öffnet diese im Browser (\textbf{/ZÄ1/}).

\subsubsection{einstellenRateAktualisierung}

 Das Programm hat Funktion die Aktualisierung der Laufwerkswerte sekündlich durchzuführen (\textbf{/ZÄ2/}).

\subsection{Neue Klasse}

Um das Bubble-Diagramm zu erzeugen wurde eine neue Klasse ''BubblePlotHandler'' (Abbildung \ref{fig:BPH}) angelegt. Die Klasse besitzt eine public Methode makeHTML(), eine private Methode getDataFromFile(path) und keine Attribute.
\\

\begin{figure}[H]
	\centering
	\includegraphics[width=1.0\textwidth]{Bilder/BubblePlotHandler}
	\caption{BubblePlotHandler.h}
	\label{fig:BPH}
\end{figure}\
\\
Wird makeHTML() aufgerufen, werden Daten aus CSV-Dateien ausgelesen (siehe Abbildung \ref{fig:CSVEx}). Wir konnten nicht herausfinden wie der Pfad zu den Festplattendaten auszulesen ist, deshalb liest die Klasse Beispieldaten aus, die im selben Ordner liegen (Smart/testData).\footnote{Die Daten liegen als ZIP-Archiv (testData.zip) bei und müssen manuell eingefügt werden, da Crystal Disk Info den Smart Ordner frei räumt, wenn es auf einer neuen Maschine gestartet wird.}
\\
\begin{figure}[H]
	\centering
	\includegraphics[width=0.75\textwidth]{Bilder/DataExample}
	\caption{Beispiel für Datenreihe in CSV-Datei}
	\label{fig:CSVEx}
\end{figure}\

\subsection{Domänenmodell}

\begin{figure}[H]
	\centering
	\includegraphics[width=1.3\textwidth]{Bilder/DomaenenModell_1}
	\caption{Domänenmodell: Funktionsweise nach der Änderung}
	\label{fig:Dom2}
\end{figure}  


\section{Beschreibung der Implementierung}

Um das Bubble-Chart darzustellen wurde, wie im Originalprogramm, eine JavaScript Bibliothek genutzt. Es wurde sich allerdings für eine andere, modernere Bibliothek entschieden, da die in Crystal Disk Info verwendete schon sehr alt war und die Anforderung ein Bubble-Chart darzustellen nicht erfüllen konnte. Deshalb wurde die JavaScript Bibliothek ChartsBundle.js  verwendet, welche auch OpenSource auf GitHub.com zur Verfügung steht.
\\\\
Mithilfe dieser Bibliothek wurde dann eine HTML-Datei mit JavaScript erstellt. Um die Verarbeitung in C++ zu vereinfachen, wurde das Skript in zwei Teile geteilt und in Text-Dateien abgelegt. Diese Dateien wurden dann in C++ mit den ausgelesen Daten zu einer fertigen HMTL/JS-Datei verknüpft und gespeichert. Die erstellte HTML Datei wird nun über einen 
Aufruf im Browser geöffnet.

\section{Programm nach Änderung}

Die implementierten Änderungen, sind auf den ersten Blick in Abbildung 10 nicht sofort ersichtlich. Das hinzufügen der Aktualsierungsrate der Festplattenwerte mit dem Wert von einer Sekunde, ist unter dem Menüpunkt Ansicht auswählbar. Die neue Diagrammfunktion, verdrängte nicht die schon vorher verfügfbaren Darstellungen. Das neue Bubble-Diagramm existiert nun parallel zu der zuvor genannten Diagrammfunktion.
\\\\
Um diese Darstellung aufrufen zu können, ist ebenfalls der Menüpunkt Ansicht anzuwählen.
Der Nutzer hat nun unter einem Unterpunkt namens Charts, die Möglichkeit diese neue Funktion auszuwählen und sich darstellen zu lassen. Auf Abbildung 11 ist diese neue Option sichtbar.


\begin{figure}[H]
	\centering
	\includegraphics [width=0.85\textwidth]{Bilder/after} 
	\caption{Screenshot Crystal Disk Info nach Änderung}
	\label{fig:Screenshot2}
\end{figure}\

\begin{figure}[H]
	\centering
	\includegraphics [width=0.95\textwidth]{Bilder/bubbleChart} 
	\caption{Neuer Diagrammtyp "Bubble Diagramm"}
	\label{fig:Screenshot3}
\end{figure}\

\section{Fazit}

Schwierigkeiten traten im gesamten Verlauf zur Genüge auf. Zuerst war recht schwer zu lokalisieren, wie Crystal Disk Info aus dem Source Code funktionsfähig zu bekommen ist. Nötig waren dafür Dateien von der fertigen Installation die dem Source Code nicht beilagen und auch nicht beim Compilieren erstellt wurden.
\\\\
Als nächstes war es sehr schwer den Code zu verstehen, da es sich bei dem Entwickler von Crystal Disk Info offensichtlich um einen sehr erfahrenen Programmierer handelt. Weiterhin ist Crystal Disk Info sehr spärlich kommentiert in den Hauptklassen.
Viele Kommentare waren auf Japanisch was die Bearbeitung verzögerte, da wir diese erst übersetzen mussten.  
\\
Die geforderte Änderung umzusetzen erwies sich als schwerer als erwartet. Wir versuchten erst zu verstehen, wie Miyazaki den Graph erstellte und stießen dabei auf eine JavaScript Bibliothek. Allerdings war auch diese Bibliothek nicht zielführend. Wir suchten uns dann einige Bibliotheken für C++ heraus, konnten aber keine verwenden, da etwaige andere Abhängigkeiten zu weiteren Bibliotheken bestanden, wie zum Beispiel zu QT (einer konkurrierenden GUI-Bibliothek zu MFC) oder zu Python. 
\\\\
Letztendlich verwendeten wir eine kleine JavaScript Bibliothek. Nur fanden wir dann heraus, dass MFC keine Möglichkeit hat JavaScript in HTML zu verarbeiten. Also bauten wir intern eine HTML und riefen sie im Standardbrowser auf.
\\\\
Durch diese vielen Probleme lernten wir, dass viele Details beim Softwaredesign zu beachten sind. Des Weiteren war eine genau Planung erforderlich. Schlussendlich wurde uns bewusst, dass sich initial einfach erscheinende Änderungen im Endeffekt aufwendiger als gedacht herausstellen können. 

\end{document}